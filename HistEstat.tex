\documentclass[a4paper,12pt]{article}
\input{abnt}
\begin{document}
\begin{center}
\huge{\textit{\textbf{Estatística ontem e hoje}}}
\end{center}
Assim como a probabilidade é difícil estabelecer quando e onde extamente surge a estatística. Até porque, provavelmente, seu surgimento ocorreu simultâneamente em  diferentes regiões. Além disso, a estatística usada pelos pioneiros nessa área, certamente, não é como conhecemos hoje.\\
Entre os dados antigos mais mencionados está o do filósofo Confúcio que relatou sobre os levantamentos realizados na China por meio vias estatísticas, antes mesmo da era cristã. Mais do que isso, também há evidência de uso informações de caráter estatítico pelos faraós do antigo Egito.\\
Todavia, a origem da estatística se deu, de forma geral e mais ampla no Renascimento, devido a pressões e necessidades de coletas de dados do Estado. Nesse contexto, manusear e analisar informações importantes  da populaçã e da economiavpara a Igreja Católica e  especialmente para governantes ficou cada vez mais alta. Tendo vista deter mais conhecimento sobre as nações e, assim, promover planos governamentais, principalmente  com fins economicos e militares, alguns nomes começaram a se destacar  como nesta ciência exata como Gottfried Achenwal que cunhou o termo \emph{estatística}.\\
\hline
\begin{center}
    Primeira tábula de vida
\end{center}
Um dos famosos dados de estatística foi de um britânico denominado Graunt. Ele reuniu uma serie de dados de 1604 a 1660 nas oaróquias de Londres, sobre os nascimentos e chegou a conclusão de que havia maior nascimento de pessoas de sexo masculino do que feminino, porém havia uma distribuição aproximadamente igual na população entre os dois sexos, devido a maior taxa de mortalidade do sexo masculino.\\
\hline
\begin{center}
    \textbf{? O que é a estatística hoje?}
\end{center}

 Atualmente é difícil encontrar uma área de estudo da ciência que não venha a usa da estística em algum nível, pois o controle de informações e soluções de diversos problemas são facilitados com o auxilio da estatística.
 Nesse sentido, foi obsevado em 1970, em uma conferência, a importância de acrescentar no curriculo escolar o tópico de estatística nas aulas de matemática, resaltando, dessa forma, a relevância da estatística na sociedade.
 
 Hoje, com as bases mais solidificadas, pode-se dividir a estatística em duas partes, a decritiva e a inferencial, sendo a que a primeira está vinculada com números, isso é, fronece resumo conciso dos dados por meio de formas númericas ou gráficas, enquanto a segunda está mais relacionada com métodos de estimativas  com base em amostras aleatórias coletadas para realiazar inferências sobre o assunto estudado.
 
 Mais do que isso, a estatística tem se mostrado muito eficiente quando aliada a computação. Os modelos estatísticos antigos eram, em sua maioria, lineares e muito simplificados. Porém, os modelos mais modernos, juntamente com algoritmos númericos expendiram a área de modelos não lineares e multi-nível, impulsionando e expandindo ainda mais a atuação da estatistica na vida cotidiana.\\
 
 $\bowtie$ \textbf{\textit{Referências}}
 
 $\longrightarrow$ https://notasdeaula.files.wordpress.com/2009/08/estatistica-e-sua-historia.pdf
 
 $\longrightarrow$ 
 https://www.ime.usp.br/~rvicente/JMPMemoria_Historia_Estatistica.pdf
 
 $\longrightarrow$ 
 https://pt.wikipedia.org/wiki/Estat%C3%ADstica
 
 $\longrightarrow$ 
 https://www.todamateria.com.br/estatistica-conceito-fases-metodo/
\end{document}
